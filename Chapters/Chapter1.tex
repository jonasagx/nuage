\chapter{Algoritmos de extração e reconhecimento de características}

Em visão computacional e processamento de imagem o conceito de detecção de
características (Feature detenction do termo em inglês) se refere aos
métodos que procuram através da extração de informações de uma
imagem tomar decisões em cada ponto de uma imagem: se existe uma característica
de algum tipo nesse ponto ou não. As características resultantes serão
subconjuntos do domínio da imagem, frequentemente na forma de pontos isolados,
curvas contínuas ou regiões conectadas.

First, ‘interest points’ are selected at distinctive locations in the image,
such as corners, blobs, and T-junctions. The most valuable property of an
interest point detector is its repeatability, i.e. whether it reliably finds
the same interest points under different viewing conditions. Next, the
neighbourhood of every interest point is represented by a feature vector. This
descriptor has to be distinctive and, at the same time, robust to noise,
detection errors, and geometric and photometric deformations. Finally, the
descriptor vectors are matched between different images. The matching is often
based on a distance between the vectors, e.g. the Mahalanobis or Euclidean
distance. The dimension of the descriptor has a direct impact on the time this
takes, and a lower number of dimensions is therefore desirable

Primeiro, pontos de interesse são selecionados em locais distintos na imagem,
tais como bordas, blocos e junções em formato de T. A propriedade mais valiosa
de um detector de pontos de interesse é sua repetibilidade, isso é: se é capaz
de encontrar os mesmos pontos de interesse sob diferentes condições. Em
seguida, a vizinhaça de cada ponto de interesse é representada por um vetor de
características. Esse descritor deve ser distintivo, e ao mesmo tempo,
resistente ao ruído, detecção de erros, e deformações geométricas e
fotométricas. Por fim, os vetores de caractéricas são combinados entre as
imagens, essa associação é geralmente baseada na distância entre os vetores,
como distância de Mahalanobis ou de Euclides. A dismensão do descritor tem um
impacto direto no tempo para realizar essa associação, e um baixo número de
dimensões é assim desejado.

\section{SURF - Speeded up robust features}
Here, we focus on scale and image rotation invari ant detectors and
descriptors. These seem to offer a good compromise between feature complexity
and robustness to commonly occurring deformations. Skew, anisotropic scaling,
and perspective effects are assumed to be second-order effects, that are
covered to some degree by the overall robustness of the descriptor. As also
claimed by Lowe [2], the additional complexity of full affine-invariant
features often has a negative impact on their robustness and does not pay off,
unless really large viewpoint changes are to be expected. In some cases, even
rotation invariance can be left out, resulting in a scale-invariant only
version of our descriptor, which we refer to as ’upright SURF’
(U-SURF). Indeed, in quite a few applications, like mobile robot navigation or
visual tourist guiding, the camera often only rotates about the vertical
axis. The benefit of avoiding the overkill of rotation invariance in such cases
is not only increased speed, but also increased discriminative
power. Concerning the photometric deformations, we assume a simple linear model
with a scale factor and offset. Notice that our detector and descriptor don’t use colour.

\chapter{Experimento comparativo}

Uma das principais dificuldades da análise do trajeto das nuvens está na comparação dos resultados, algoritmos automáticos servem para agilizar e dar precisão ao processo. Na década de 60, nos primeiros experimentos do gênero, os mapas eram gerados manualmente pela comparação da posição das nuvens, à olho humano, da série histórica de fotos de uma região. Essa abordagem consome tempo gasto, quanto mais preciso e/ou quanto maior a área do mapa a ser gerado.

Este trabalho propõe então uma forma de comparação dentro de um pequeno contexto, a comparação de diferentes algoritmos de uma mesma categoria (detecção de características), no que diz respeito à extadião do mapa e o uso de uma métrica chamada aqui de \textit{convergência}.
